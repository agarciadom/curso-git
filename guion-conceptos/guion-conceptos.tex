%; whizzy -pdf

\documentclass[a4paper,11pt]{article}
\usepackage[latin1]{inputenc}
\usepackage[spanish]{babel}
\usepackage{amsthm}
\usepackage{graphicx}
\usepackage{hyperref}

\title{Gui�n de pr�cticas:\\Conceptos b�sicos}
\author{Antonio Garc�a Dom�nguez}
\date{\today}

\RequirePackage{amsthm}

\newtheorem{pregunta}{Pregunta}
\newcounter{saveenum}
\newcommand{\savelp}{\setcounter{saveenum}{\value{enumi}}}
\newcommand{\loadlp}{\setcounter{enumi}{\value{saveenum}}}
\newenvironment{first-enumerate}{\begin{enumerate}}{\savelp{}\end{enumerate}}
\newenvironment{continue-enumerate}{\begin{enumerate}\loadlp{}}{\savelp{}\end{enumerate}}

%%% Local Variables: 
%%% mode: latex
%%% TeX-master: t
%%% End: 


\begin{document}

\maketitle
\begin{center}
  Distribuido bajo la licencia CC v3.0 BY-SA (\url{http://creativecommons.org/licenses/by-sa/3.0/deed.es}).

  \vskip 2em

  \includegraphics{cc-by-sa}
\end{center}

\vskip .1\textheight

\tableofcontents{}

\clearpage

\section{Clonar un repositorio existente}
\label{sec:clon-repos}

Clonar.

\section{Navegar por el historial}
\label{sec:navegar-historial}

\section{Examinar los objetos a bajo nivel}
\label{sec:examinar-objetos}




\end{document}

%%% Local Variables: 
%%% mode: latex
%%% TeX-master: t
%%% End: 
