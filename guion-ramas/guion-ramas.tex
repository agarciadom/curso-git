%; whizzy -pdf

\documentclass[a4paper,11pt]{article}
\usepackage[latin1]{inputenc}
\usepackage[spanish]{babel}
\usepackage{amsthm}
\usepackage{graphicx}
\usepackage{hyperref}

\title{Gui�n de pr�cticas:\\Gesti�n de ramas\footnote{Sin preguntas por el momento.}}
\author{Antonio Garc�a Dom�nguez}
\date{\today}

\RequirePackage{amsthm}

\newtheorem{pregunta}{Pregunta}
\newcounter{saveenum}
\newcommand{\savelp}{\setcounter{saveenum}{\value{enumi}}}
\newcommand{\loadlp}{\setcounter{enumi}{\value{saveenum}}}
\newenvironment{first-enumerate}{\begin{enumerate}}{\savelp{}\end{enumerate}}
\newenvironment{continue-enumerate}{\begin{enumerate}\loadlp{}}{\savelp{}\end{enumerate}}

%%% Local Variables: 
%%% mode: latex
%%% TeX-master: t
%%% End: 


\begin{document}

\maketitle
\begin{center}
  Distribuido bajo la licencia CC v3.0 BY-SA (\url{http://creativecommons.org/licenses/by-sa/3.0/deed.es}).

  \vskip 2em

  \includegraphics{../cc-by-sa}
\end{center}

\vskip .1\textheight

\tableofcontents{}

\clearpage

\section{Gesti�n b�sica de ramas}
\label{sec:creacion-ramas}

% git branch
% git branch -a
% git branch -d
% git branch -D
% gitk
% gitk --all

\section{Cambio entre ramas}
\label{sec:cambio-entre-ramas}

% git checkout
% git checkout -b <commitish>
% mencionar que podemos ir a algo que no sea la punta de una rama, pero los problemas que trae

% git stash (v1.5.3+)
% git stash apply
% git stash drop
% git stash pop
% git stash list
% git stash clear

\section{Reuni�n de ramas}
\label{sec:reunion-ramas}

% git merge
%
% conflictos:
% * git diff --ours
% * git diff --theirs
% * git diff --base
% * resoluci�n

\section{Cambiar la revisi�n base de una rama}
\label{sec:cambiar-rev-base-rama}

% git rebase
% git rebase --interactive


\end{document}

%%% Local Variables: 
%%% mode: latex
%%% TeX-master: t
%%% End: 
