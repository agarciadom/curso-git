\documentclass[compress,xcolor=svgnames]{beamer}

\mode<presentation>
{
  \useoutertheme[subsection=false,footline=authorinstitutetitle]{miniframes}
  \usecolortheme{whale}
  \usecolortheme{orchid}
  \useinnertheme{rounded}
  \setbeamertemplate{navigation symbols}{}
  \setbeamercovered{dynamic}
}

\usepackage[spanish,es-noshorthands]{babel}
\usepackage[utf8x]{inputenc}
\usepackage[T1]{fontenc}
\usepackage{csquotes}
\usepackage{tikz}
\usepackage{fancyvrb}
\usepackage[htt]{hyphenat}

\PrerenderUnicode{áéíóúÁÉÍÓÚçÇ}

\title{Introducción al Sistema de Control de Versiones Distribuido Git}
\author{Antonio García Domínguez}
\date{\today}
\institute{Universidad de Cádiz}

\AtBeginSubsection[]
{
  \begin{frame}<beamer>{Contenidos}
    \tableofcontents[currentsection,currentsubsection]
  \end{frame}
}

\usetikzlibrary{positioning,shapes}

% Semantic formatting
\newcommand*{\paquete}[1]{\texttt{#1}}
\newcommand*{\fichero}[1]{\textit{#1}}
\newcommand*{\rama}[1]{\structure{#1}}
\newcommand*{\tipo}[1]{\textit{#1}}

% For formatting commands
\newcommand*{\orden}[1]{
  {\small\ttfamily\$~\nohyphens{#1}\\}}

% For running Git commands and converting the ANSI escape sequences to LaTeX
\newcommand{\runcommand}[1]{\immediate\write18{#1}}
\newcommand{\showcommand}[2][cat]{
  \runcommand{(#2) | ansifilter -Lf | #1 | tee cmd.tmp}
  {\small\ttfamily\input{cmd.tmp}}
  \runcommand{rm -f cmd.tmp}}
\newcommand{\runandshowcommand}[1]{
  \orden{#1}
  \showcommand{#1}}

% For the diagrams talking about the existing types of VCS
\newenvironment{vcstypes}{
   \begin{tikzpicture}[
      node distance=6em,
      every path/.style={very thick},
      repo/.style={draw,rounded corners,fill=gray!50},
      wcopy/.style={draw,rounded corners,fill=green!20}]
}{\end{tikzpicture}}

\begin{document}

\begin{frame}
  \titlepage
\end{frame}

\begin{frame}{Contenidos}
  \tableofcontents
\end{frame}

\section{Introducción}

\begin{frame}{¿Por qué usar un SCV?}

  \begin{block}{Copiar ficheros y mandar correos no escala}
    \begin{itemize}
    \item ¿Cuál era la última versión?
    \item ¿Cómo vuelvo a la anterior?
    \item ¿Cómo reúno mis cambios con los de otro?
    \end{itemize}
  \end{block}

  \begin{block}{SCV: todo ventajas a cambio de alguna disciplina}
    \begin{itemize}
    \item Llevamos un historial de los cambios
    \item Podemos ir trabajando en varias cosas a la vez
    \item Podemos colaborar con otros
    \item Hacemos copia de seguridad de todo el historial
    \end{itemize}
  \end{block}
\end{frame}

\begin{frame}{Historia de los SCV}

  \begin{block}{Sin red, un desarrollador}
    \begin{description}
    \item[1972] Source Code Control System
    \item[1980] Revision Control System
    \end{description}
  \end{block}

  \begin{block}{Centralizados}
    \begin{description}
    \item[1986] Concurrent Version System
    \item[1999] Subversion (\enquote{CVS done right})
    \end{description}
  \end{block}

  \begin{block}{Distribuidos}
    \begin{description}
    \item[2001] Arch, monotone
    \item[2002] Darcs
    \item[2005] Git, Mercurial (hg), Bazaar (bzr)
    \end{description}
  \end{block}

\end{frame}

\begin{frame}{Historia de Git}

  \begin{block}{Antes de BitKeeper}
    Para desarrollar Linux, se usaban parches y \texttt{tar.gz}.
  \end{block}

  \begin{block}{BitKeeper}
    \begin{description}
    \item[02/2002] BitMover regala licencia BitKeeper (privativo)
    \item[04/2005] BitMover retira la licencia tras roces
    \end{description}
  \end{block}

  \begin{block}{Git}
    \begin{description}
    \item[04/2005] Linus Torvalds presenta Git, que ya reúne ramas
    \item[06/2005] Git se usa para gestionar Linux
    \item[02/2007] Git 1.5.0 es utilizable por mortales
    \item[12/2010] Última versión: Git 1.7.3.2
    \end{description}
  \end{block}

\end{frame}

\begin{frame}{SCV centralizados}

  \begin{center}
    \begin{vcstypes}
      \draw[white] (3,3) rectangle (-3, -3);
      \node[repo]  (r)  {Repositorio central};

      \node<2->[wcopy,above right of=r] (w1) {Desarrollador A};
      \draw<2>[->,color=DarkGreen] (r) edge node[midway,right] {checkout} (w1);
      \draw<3>[<-,red] (r) edge node[midway,right] {commit} (w1);
      \draw<4->[-] (r) edge (w1);

      \node<2->[wcopy,below right of=r] (w2) {Desarrollador B};
      \draw<2>[->,color=DarkGreen] (r) edge node[midway,right] {checkout} (w2);
      \draw<3,5>[-] (r) edge (w2);
      \draw<4>[->,blue] (r) edge node[midway,right] {update} (w2);
    \end{vcstypes}
  \end{center}

  \begin{overprint}
    \onslide<1>
    \centering
    Tenemos nuestro repositorio central con todo dentro.

    \onslide<2>
    \centering
    Los desarrolladores crean \alert{copias de trabajo}.

    \onslide<3>
    \centering
    El desarrollador A manda sus cambios al servidor.

    \onslide<4>
    \centering
    El desarrollador B los recibe.

    \onslide<5>
    \centering ¿Y si se cae el servidor, o la red?
  \end{overprint}

\end{frame}

\begin{frame}{SCV distribuidos}

  \begin{center}
    \begin{vcstypes}
      \draw[white] (5,2) rectangle (-5, -2);

      \node[repo] (ra) at ( 0,-1) {Repositorio A};

      \node<2->[repo] (rb) at ( 3, 1) {Repositorio B};
      \draw<2>[->,color=DarkGreen] (ra) edge node[midway,right] {clone} (rb);
      \draw<3>[->,color=blue] (ra) edge node[midway,right] {pull} (rb);
      \draw<4>[<-,color=red] (ra) edge node[midway,right] {push} (rb);

      \node<5->[repo] (rc) at (-3, 1) {Repositorio C};
      \draw<5>[->,color=DarkGreen] (ra) edge node[midway,left] {clone} (rc);
      \draw<6>[<-,color=red] (ra) edge node[midway,left] {push} (rc);
      \draw<7>[<-,color=red] (ra) edge node[draw,fill=white,midway] {X} (rc);

      \draw<8>[->,color=red] (rc) edge node[midway,above] {push} (rb);

    \end{vcstypes}
  \end{center}

  \begin{overprint}
    \onslide<1> \centering Tenemos nuestro repositorio.

    \onslide<2> \centering Alguien \alert{clona} el repositorio.

    \onslide<3> \centering De vez en cuando se trae nuestros cambios recientes.

    \onslide<4> \centering De vez en cuando nos manda sus cambios.

    \onslide<5> \centering Viene otro desarrollador.

    \onslide<6> \centering Intenta hacer sus cambios locales...

    \onslide<7> \centering Pero no le funciona, o no tiene permisos para ello.

    \onslide<8> \centering Se los pasa al otro desarrollador sin más.

    \onslide<9> \centering La diferencia entre los repositorios es
    \emph{social}, no técnica.
  \end{overprint}

\end{frame}

\begin{frame}{Ventajas de un SCV distribuido (I)}

  \begin{block}{Rapidez}
    \begin{itemize}
    \item Todo se hace en local: el disco duro es más rápido que la
      red, y cuando esté todo en caché será más rápido aún
    \item Clonar un repositorio Git suele tardar \emph{menos} que
      crear una copia de trabajo de SVN, y ocupa menos
    \end{itemize}
  \end{block}

  \pause

  \begin{block}{Revisiones pequeñas y sin molestar}
    \begin{itemize}
    \item Nadie ve nada nuestro hasta que lo mandamos
    \item Podemos ir haciendo revisiones pequeñas intermedias
    \item Sólo mandamos cuando compila y supera las pruebas
    \item Podemos hacer experimentos de usar y tirar
    \end{itemize}
  \end{block}

\end{frame}

\begin{frame}{Ventajas de un SCV distribuido (II)}
  \begin{block}{Trabajo sin conexión}
    \begin{itemize}
    \item En el tren, avión, autobús, etc.
    \item Aunque no tengamos permisos de escritura
    \item Aunque se caiga la red, se puede colaborar
    \end{itemize}
  \end{block}

  \pause

  \begin{block}{Robustez}
    Falla el disco duro del repositorio bendito. ¿Qué hacer?
    \begin{itemize}
    \item Centralizado: copias de seguridad
    \item Distribuido: copias de seguridad y/o colaborar por otros medios
    \end{itemize}
  \end{block}
\end{frame}

\section[Local]{Trabajando en local}

\subsection{Preparaciones}

\begin{frame}{Instalación de Git}
  \begin{block}{Ubuntu Linux}
    \begin{itemize}
    \item 9.10: descargar paquete de Squeeze (Debian)
    \item 10.04: instalar \paquete{git-*}
    \item 10.10: instalar \paquete{git-all}
    \item Instalad \paquete{tkdiff} (para conflictos) y un buen editor
    \item Fuentes: guión \fichero{install-git.sh} en materiales del curso
    \end{itemize}
  \end{block}

  \begin{block}{Windows}
    \begin{itemize}
    \item Usuarios: msysGit (\url{https://code.google.com/p/msysgit/})
    \item Desarrolladores: Cygwin (\url{http://www.cygwin.com/})
    \end{itemize}
  \end{block}
\end{frame}

\begin{frame}{Configuración inicial}
  \framesubtitle{Cambiamos la configuración global en \fichero{\$HOME/.gitconfig}}

  \begin{block}{Identificación}
    \orden{git config {-}-global user.name "Mi Nombre"}
    \orden{git config {-}-global user.email mi@correo}
  \end{block}

  \begin{block}{Editor: por defecto Vi/Vim}
    \orden{git config {-}-global core.editor emacs}
  \end{block}

  \begin{block}{Herramienta para resolver conflictos}
    \orden{git config {-}-global merge.tool tkdiff}
  \end{block}

\end{frame}

\subsection{Uso básico}

% Limpiamos el área de trabajo
\runcommand{rm -rf ejemplo}

\begin{frame}{Creación de un repositorio}
  Sólo tenemos que ir a un directorio y decirle a Git que cree un
  repositorio ahí.

  \vfill

  \runandshowcommand{mkdir ejemplo}
  \orden{cd ejemplo}
  \orden{git init}
  \showcommand{cd ejemplo; git init | fold -sw 60; git config user.name Antonio; git config user.email a@b.com; git config color.ui always}
\end{frame}

\begin{frame}{Nuestras dos primeras revisiones en la rama master}

  \orden{echo "hola" > f.txt}
  \runcommand{echo "hola" | tee ejemplo/f.txt}
  \orden{git add f.txt}
  \showcommand{cd ejemplo; git add f.txt}
  \orden{git commit -m "primer commit"}
  \showcommand{cd ejemplo; git commit -m "primer commit"}
  \orden{echo "adios" {>}> f.txt}
  \runcommand{echo "adios" | tee -a ejemplo/f.txt}
  \orden{git add f.txt}
  \showcommand{cd ejemplo; git add f.txt}
  \orden{git commit -m "segundo commit"}
  \showcommand{cd ejemplo; git commit -m "segundo commit"}

  \vfill

  \begin{itemize}
  \item ¿Qué es ese número extraño después de \enquote{root-commit}?
  \item ¿Dónde están mis revisiones?
  \item ¿Por qué hemos hecho \texttt{git add} dos veces?
  \end{itemize}
\end{frame}

\begin{frame}{Modelo de datos de Git}

  \begin{block}{Idea central}
    4 tipos de objetos direccionables por contenido (resumen SHA1)
  \end{block}

  \vspace{-1em}

  \begin{overprint}
    \onslide<1>
    \begin{block}{Revisiones (\tipo{commits})}
      \begin{itemize}
      \item Fecha, hora, autoría, fuente y un mensaje
      \item Referencia a revisión padre y a un \tipo{tree}
      \end{itemize}
    \end{block}

    \orden{git cat-file -p HEAD}
    \showcommand{cd ejemplo; git cat-file -p HEAD}

    \onslide<2>
    \begin{block}{Árboles (\tipo{trees})}
      Lista \tipo{blobs} y \tipo{trees}, dándoles nombres
    \end{block}

    \orden{git cat-file -p HEAD:}
    \showcommand{cd ejemplo; git cat-file -p HEAD:}

    \onslide<3>
    \begin{block}{Ficheros normales (\tipo{blobs})}
      Secuencias de bytes sin ningún significado particular
    \end{block}

    \orden{git cat-file -p HEAD:f.txt}
    \showcommand{cd ejemplo; git cat-file -p HEAD:f.txt}

    \onslide<4>
    \begin{block}{Etiquetas (\tipo{tags})}
      Referencias simbólicas inmutables a objetos (p.ej. \tipo{commits})
    \end{block}

    \orden{git tag -a v1.0 -m "version 1.0" HEAD}
    \runcommand{cd ejemplo; git tag -a v1.0 -m "version 1.0" HEAD}
    \orden{git cat-file -p v1.0}
    \showcommand{cd ejemplo; git cat-file -p v1.0}
  \end{overprint}

\end{frame}

\begin{frame}{Estructura de un repositorio Git}

  \begin{itemize}
  Todos va a \fichero{.git} bajo la raíz del repositorio


  que no es

el \emph{repositorio},

  Un repositorio Git no es más que un directorio con un par de
  ficheros. La mayoría están en texto plano.

  \vfill

  \orden{ls .git}
  \showcommand{cd ejemplo; ls --color=always .git}
\end{frame}

\begin{frame}[t]{Algunos de los ficheros en \fichero{.git}}

  \begin{overlayarea}{\textwidth}{2.5cm}
    \only<1>{
      \begin{block}{config}
        Contiene la configuración local.
      \end{block}
    }
    \only<2>{
      \begin{block}{description}
        Descripción corta textual para \paquete{gitweb}.
      \end{block}
    }
    \only<3>{
      \begin{block}{HEAD}
        Referencia simbólica a la revisión sobre la que estamos
        trabajando.
      \end{block}
    }
    \only<4>{
      \begin{block}{hooks}
        Manejadores de eventos. Ahora sólo tenemos ejemplos.
      \end{block}
    }
    \only<5>{
      \begin{block}{info/exclude}
        Patrones de ficheros a ignorar.
      \end{block}
    }
    \only<6>{
      \begin{block}{objects}
        Objetos del repositorio, en formato suelto justo dentro de
        \fichero{objects} o empaquetado en \fichero{pack}. Ahora mismo
        no hay.
      \end{block}
    }
    \only<7>{
      \begin{block}{refs}
        Referencias simbólicas a las puntas de cada rama y a las
        etiquetas privadas. Ahora mismo no hay.
      \end{block}
    }
  \end{overlayarea}

  \begin{overlayarea}{\textwidth}{6cm}
    \only<1>{
      \orden{cat config}
      \showcommand{cat ejemplo/.git/config | sed -n '1,/logallrefupdates/p'}
    }
    \only<2>{
      \orden{cat description}
      \showcommand{cat ejemplo/.git/description | fold -sw 60}
    }
    \only<3>{
      \orden{cat HEAD}
      \showcommand{cat ejemplo/.git/HEAD}
    }
    \only<4>{
      \orden{ls hooks | head -5}
      \showcommand{ls --color=always ejemplo/.git/hooks | head -5}
    }
    \only<5>{
      \orden{cat info/exclude}
      \showcommand{cat ejemplo/.git/info/exclude | ./shorten.sh}
    }
    \only<6>{
      \orden{ls -r objects}
      \showcommand{ls -r --color=always ejemplo/.git/objects}
    }
    \only<7>{
      \orden{ls -r refs}
      \showcommand{ls -r --color=always ejemplo/.git/refs}
    }
  \end{overlayarea}

\end{frame}

\begin{frame}{Área de preparación}

\end{frame}

\begin{frame}{Conseguir ayuda}

\end{frame}

\section[Centralizado]{Flujo de trabajo centralizado}

\subsection{De Git a Git}

\begin{frame}{Trabajar con Git}

\end{frame}

\subsection{De Git a SVN}

\begin{frame}{Interoperar con SVN}

\end{frame}

\section[Distribuido]{Flujo de trabajo distribuido}

\begin{frame}{Colaborando entre varios repositorios}

\end{frame}

\section[Avanzado]{Aspectos avanzados}

\begin{frame}{Esculpir revisiones con add -i}

\end{frame}

\begin{frame}{Replantear ramas con rebase}

\end{frame}

\begin{frame}{Reorganizar ramas con rebase -i}

\end{frame}

\appendix

\begin{frame}{Fin de la presentación}
  \begin{center}
    {\Huge ¡Gracias por su atención!}

    \vspace{3em}

    {\Large
      \href{mailto:antonio.garciadominguez@uca.es}{antonio.garciadominguez@uca.es}}
  \end{center}
\end{frame}

\end{document}
