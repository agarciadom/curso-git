\documentclass[xcolor=svgnames]{beamer}

\mode<presentation>
{
  \useoutertheme[subsection=false,footline=authorinstitutetitle]{miniframes}
  \usecolortheme{whale}
  \usecolortheme{orchid}
  \useinnertheme{rounded}
  \setbeamertemplate{navigation symbols}{}
  \setbeamercovered{dynamic}
}

\usepackage[spanish,es-noshorthands]{babel}
\usepackage[utf8x]{inputenc}
\usepackage[T1]{fontenc}
\usepackage{csquotes}
\usepackage{tikz}
\usepackage{fancyvrb}
\usepackage[htt]{hyphenat}

\PrerenderUnicode{áéíóúÁÉÍÓÚçÇ}

\title{Introducción al Sistema de Control de Versiones Distribuido Git}
\author{Antonio García Domínguez}
\date{\today}
\institute{Universidad de Cádiz}

\AtBeginSection[]
{
  \begin{frame}<beamer>{Contenidos}
    \tableofcontents[currentsection,hideothersubsections]
  \end{frame}
}

\AtBeginSubsection[]
{
  \begin{frame}<beamer>{Contenidos}
    \tableofcontents[currentsection,subsectionstyle=show/shaded/hide]
  \end{frame}
}

\usetikzlibrary{positioning,shapes}

% Semantic formatting
\newcommand*{\paquete}[1]{\texttt{#1}}
\newcommand*{\fichero}[1]{\textit{#1}}
\newcommand*{\rama}[1]{\structure{#1}}
\newcommand*{\tipo}[1]{\textit{#1}}

% For formatting commands
\newcommand*{\inlinecmd}[1]{{\small\ttfamily\nohyphens{#1}}}
\newcommand*{\orden}[1]{{\scriptsize\ttfamily\$~\nohyphens{#1}\\}}

% For running Git commands and converting the ANSI escape sequences to LaTeX
\newcommand{\runcommand}[1]{\immediate\write18{#1}}
\newcommand{\showcommand}[2][cat]{%
  \runcommand{(#2) | ansifilter -Lf | #1 | tee cmd.tmp}%
  {\scriptsize\ttfamily\input{cmd.tmp}}\runcommand{rm -f cmd.tmp}}
\newcommand{\runandshowcommand}[1]{
  \orden{#1}\showcommand{#1}}

% For the diagrams talking about the existing types of VCS
\newenvironment{vcstypes}{
   \begin{tikzpicture}[
      node distance=6em,
      every path/.style={very thick},
      repo/.style={draw,rounded corners,fill=gray!50},
      wcopy/.style={draw,rounded corners,fill=green!20}]
}{\end{tikzpicture}}

\begin{document}

\begin{frame}
  \titlepage
\end{frame}

\begin{frame}{Contenidos}
  \tableofcontents[hideallsubsections]
\end{frame}

\section{Introducción}

\subsection{Antecedentes}

\begin{frame}{¿Por qué usar un SCV?}

  \begin{block}{Copiar ficheros y mandar correos no escala}
    \begin{itemize}
    \item ¿Cuál era la última versión?
    \item ¿Cómo vuelvo a la anterior?
    \item ¿Cómo reúno mis cambios con los de otro?
    \end{itemize}
  \end{block}

  \begin{block}{SCV: todo ventajas a cambio de alguna disciplina}
    \begin{itemize}
    \item Llevamos un historial de los cambios
    \item Podemos ir trabajando en varias cosas a la vez
    \item Podemos colaborar con otros
    \item Hacemos copia de seguridad de todo el historial
    \end{itemize}
  \end{block}
\end{frame}

\begin{frame}{Historia de los SCV}

  \begin{block}{Sin red, un desarrollador}
    \begin{description}
    \item[1972] Source Code Control System
    \item[1980] Revision Control System
    \end{description}
  \end{block}

  \begin{block}{Centralizados}
    \begin{description}
    \item[1986] Concurrent Version System
    \item[1999] Subversion (\enquote{CVS done right})
    \end{description}
  \end{block}

  \begin{block}{Distribuidos}
    \begin{description}
    \item[2001] Arch, monotone
    \item[2002] Darcs
    \item[2005] Git, Mercurial (hg), Bazaar (bzr)
    \end{description}
  \end{block}

\end{frame}

\begin{frame}{Historia de Git}

  \begin{block}{Antes de BitKeeper}
    Para desarrollar Linux, se usaban parches y \texttt{tar.gz}.
  \end{block}

  \begin{block}{BitKeeper}
    \begin{description}
    \item[02/2002] BitMover regala licencia BitKeeper (privativo)
    \item[04/2005] BitMover retira la licencia tras roces
    \end{description}
  \end{block}

  \begin{block}{Git}
    \begin{description}
    \item[04/2005] Linus Torvalds presenta Git, que ya reúne ramas
    \item[06/2005] Git se usa para gestionar Linux
    \item[02/2007] Git 1.5.0 es utilizable por mortales
    \item[12/2010] Última versión: Git 1.7.3.2
    \end{description}
  \end{block}

\end{frame}

\subsection{Tipos de SCV}

\begin{frame}{SCV centralizados}

  \begin{center}
    \begin{vcstypes}
      \draw[white] (3,2) rectangle (-3, -2);
      \node[repo]  (r)  {Repositorio central};

      \node<2->[wcopy,above right of=r] (w1) {Desarrollador A};
      \draw<2>[->,color=DarkGreen] (r) edge node[midway,right] {checkout} (w1);
      \draw<3>[<-,red] (r) edge node[midway,right] {commit} (w1);
      \draw<4->[-] (r) edge (w1);

      \node<2->[wcopy,below right of=r] (w2) {Desarrollador B};
      \draw<2>[->,color=DarkGreen] (r) edge node[midway,right] {checkout} (w2);
      \draw<3,5>[-] (r) edge (w2);
      \draw<4>[->,blue] (r) edge node[midway,right] {update} (w2);
    \end{vcstypes}
  \end{center}

  \begin{overprint}
    \onslide<1>
    \centering
    Tenemos nuestro repositorio central con todo dentro.

    \onslide<2>
    \centering
    Los desarrolladores crean \alert{copias de trabajo}.

    \onslide<3>
    \centering
    El desarrollador A manda sus cambios al servidor.

    \onslide<4>
    \centering
    El desarrollador B los recibe.

    \onslide<5>
    \centering ¿Y si se cae el servidor, o la red?
  \end{overprint}

\end{frame}

\begin{frame}{SCV distribuidos}

  \begin{center}
    \begin{vcstypes}
      \draw[white] (5,2) rectangle (-5, -2);

      \node[repo] (ra) at ( 0,-1) {Repositorio A};

      \node<2->[repo] (rb) at ( 3, 1) {Repositorio B};
      \draw<2>[->,color=DarkGreen] (ra) edge node[midway,right] {clone} (rb);
      \draw<3>[->,color=blue] (ra) edge node[midway,right] {pull} (rb);
      \draw<4>[<-,color=red] (ra) edge node[midway,right] {push} (rb);

      \node<5->[repo] (rc) at (-3, 1) {Repositorio C};
      \draw<5>[->,color=DarkGreen] (ra) edge node[midway,left] {clone} (rc);
      \draw<6>[<-,color=red] (ra) edge node[midway,left] {push} (rc);
      \draw<7>[<-,color=red] (ra) edge node[draw,fill=white,midway] {X} (rc);

      \draw<8>[->,color=red] (rc) edge node[midway,above] {push} (rb);

    \end{vcstypes}
  \end{center}

  \begin{overprint}
    \onslide<1> \centering Tenemos nuestro repositorio.

    \onslide<2> \centering Alguien \alert{clona} el repositorio.

    \onslide<3> \centering De vez en cuando se trae nuestros cambios recientes.

    \onslide<4> \centering De vez en cuando nos manda sus cambios.

    \onslide<5> \centering Viene otro desarrollador.

    \onslide<6> \centering Intenta hacer sus cambios locales...

    \onslide<7> \centering Pero no le funciona, o no tiene permisos para ello.

    \onslide<8> \centering Se los pasa al otro desarrollador sin más.

    \onslide<9> \centering La diferencia entre los repositorios es
    \emph{social}, no técnica.
  \end{overprint}

\end{frame}

\begin{frame}{Ventajas de un SCV distribuido (I)}

  \begin{block}{Rapidez}
    \begin{itemize}
    \item Todo se hace en local: el disco duro es más rápido que la
      red, y cuando esté todo en caché será más rápido aún
    \item Clonar un repositorio Git suele tardar \emph{menos} que
      crear una copia de trabajo de SVN, y ocupa menos
    \end{itemize}
  \end{block}

  \pause

  \begin{block}{Revisiones pequeñas y sin molestar}
    \begin{itemize}
    \item Nadie ve nada nuestro hasta que lo mandamos
    \item Podemos ir haciendo revisiones pequeñas intermedias
    \item Sólo mandamos cuando compila y supera las pruebas
    \item Podemos hacer experimentos de usar y tirar
    \end{itemize}
  \end{block}

\end{frame}

\begin{frame}{Ventajas de un SCV distribuido (II)}
  \begin{block}{Trabajo sin conexión}
    \begin{itemize}
    \item En el tren, avión, autobús, etc.
    \item Aunque no tengamos permisos de escritura
    \item Aunque se caiga la red, se puede colaborar
    \end{itemize}
  \end{block}

  \pause

  \begin{block}{Robustez}
    Falla el disco duro del repositorio bendito. ¿Qué hacer?
    \begin{itemize}
    \item Centralizado: copias de seguridad
    \item Distribuido: copias de seguridad y/o colaborar por otros medios
    \end{itemize}
  \end{block}
\end{frame}

\section[Local]{Flujo de trabajo local}

\subsection{Preparación}

\begin{frame}{Instalación de Git}
  \begin{block}{Ubuntu Linux}
    \begin{itemize}
    \item 9.10: descargar paquete de Squeeze (Debian)
    \item 10.04: instalar \paquete{git-*}
    \item 10.10: instalar \paquete{git-all}
    \item Instalad \paquete{tkdiff} (para conflictos) y un buen editor
    \item Fuentes: guión \fichero{install-git.sh} en materiales del curso
    \end{itemize}
  \end{block}

  \begin{block}{Windows}
    \begin{itemize}
    \item Usuarios: msysGit (\url{https://code.google.com/p/msysgit/})
    \item Desarrolladores: Cygwin (\url{http://www.cygwin.com/})
    \end{itemize}
  \end{block}
\end{frame}

\begin{frame}{Configuración inicial}
  \framesubtitle{Cambiamos la configuración global en \fichero{\$HOME/.gitconfig}}

  \begin{block}{Identificación}
    \orden{git config {-}-global user.name "Mi Nombre"}
    \orden{git config {-}-global user.email mi@correo}
  \end{block}

  \begin{block}{Editor: por defecto Vi/Vim}
    \orden{git config {-}-global core.editor emacs}
  \end{block}

  \begin{block}{Herramienta para resolver conflictos}
    \orden{git config {-}-global merge.tool tkdiff}
  \end{block}

  \begin{block}{Algunos alias útiles}
    \orden{git config {-}-global alias.ci commit}
    \orden{git config {-}-global alias.st status}
    \orden{git config {-}-global alias.ai "add -i"}
  \end{block}

\end{frame}

% Limpiamos el área de trabajo
\runcommand{rm -rf ejemplo}

\begin{frame}{Creación de un repositorio}
  Sólo tenemos que ir a un directorio y decirle a Git que cree un
  repositorio ahí.

  \vfill

  \runandshowcommand{mkdir ejemplo}
  \orden{cd ejemplo}
  \orden{git init}
  \showcommand{cd ejemplo; git init | fold -sw 70; git config user.name Antonio; git config user.email a@b.com; git config color.ui always}
\end{frame}

\begin{frame}{Nuestras dos primeras revisiones en la rama master}

  \orden{echo "hola" > f.txt}
  \runcommand{echo "hola" | tee ejemplo/f.txt}
  \orden{git add f.txt}
  \showcommand{cd ejemplo; git add f.txt}
  \orden{git commit -m "primer commit"}
  \showcommand{cd ejemplo; git commit -m "primer commit"}
  \orden{echo "adios" {>}> f.txt}
  \runcommand{echo "adios" | tee -a ejemplo/f.txt}
  \orden{git add f.txt}
  \showcommand{cd ejemplo; git add f.txt}
  \orden{git commit -m "segundo commit"}
  \showcommand{cd ejemplo; git commit -m "segundo commit"}

  \vfill

  \begin{itemize}
  \item ¿Qué es ese identificador después de \enquote{root-commit}?
  \item ¿Dónde se guardan mis cosas?
  \item Usuario de SVN: \enquote{¿\texttt{git add} dos veces?}
  \end{itemize}
\end{frame}

\subsection{Conceptos}

%% MODELO DE DATOS

\begin{frame}{Modelo de datos de Git}
  \begin{block}{Características}
    \begin{itemize}
    \item Un repositorio es un grafo orientado acíclico de objetos
    \item Hay 4 tipos de objetos: \tipo{commit}, \tipo{tree},
      \tipo{blob} y \tipo{tag}
    \item Los objetos son direccionables por contenido (resumen SHA1)
    \end{itemize}
  \end{block}

  \begin{block}{Consecuencias del diseño}
    \begin{itemize}
    \item Los objetos son inmutables: al cambiar su contenido, cambia
      su SHA1
    \item Git \emph{no} gestiona información de ficheros movidos y demás
    \item Git \emph{nunca} guarda más de un objeto una vez en el DAG,
      aunque aparezca en muchos sitios
    \end{itemize}
  \end{block}
\end{frame}

\begin{frame}{Modelo de datos de Git: revisiones (\tipo{commits})}
  \begin{block}{Contenido}
    \begin{itemize}
    \item Fecha, hora, autoría, fuente y un mensaje
    \item Referencia a revisión padre y a un \tipo{tree}
    \end{itemize}
  \end{block}

  \vfill

  \orden{git cat-file -p HEAD}
  \showcommand{cd ejemplo; git cat-file -p HEAD}
\end{frame}

\begin{frame}{Modelo de datos de Git: árboles (\tipo{trees})}
  \begin{block}{Contenido}
    \begin{itemize}
    \item Lista de \tipo{blobs} y \tipo{trees}
    \item Separa el nombre de un fichero/directorio de su contenido
    \item Sólo gestiona los bits de ejecución de los ficheros
    \item No se guardan directorios vacíos
    \end{itemize}
  \end{block}

  \vfill

  \orden{git cat-file -p HEAD:}
  \showcommand{cd ejemplo; git cat-file -p HEAD:}
\end{frame}

\begin{frame}{Modelo de datos de Git: ficheros (\tipo{blobs})}
  \begin{block}{Contenido}
    Secuencias de bytes sin ningún significado particular.
  \end{block}

  \vfill

  \orden{git cat-file -p HEAD:f.txt}
  \showcommand{cd ejemplo; git cat-file -p HEAD:f.txt}
\end{frame}

\begin{frame}{Modelo de datos de Git: etiquetas (\tipo{tags})}
  \begin{block}{Contenido}
    \begin{itemize}
    \item Referencias simbólicas inmutables a otros objetos
    \item Normalmente apuntan a \tipo{commits}
    \item Pueden firmarse mediante GnuPG, protegiendo la integridad de
      todo el historial hasta entonces
    \end{itemize}
  \end{block}

  \vfill

  \orden{git tag -a v1.0 -m "version 1.0" HEAD}
  \runcommand{cd ejemplo; git tag -a v1.0 -m "version 1.0" HEAD}
  \orden{git cat-file -p v1.0}
  \showcommand{cd ejemplo; git cat-file -p v1.0}
\end{frame}

%%% MODELO DE DATOS

\begin{frame}{Estructura física de un repositorio Git}
  \begin{block}{Partes de un repositorio Git}
    \begin{itemize}
    \item Directorio de trabajo
    \item Grafo de objetos: \fichero{.git}
    \item Área de preparación: \fichero{.git/index}
    \end{itemize}
  \end{block}

  \vfill

  \orden{ls .git}
  \showcommand{cd ejemplo; ls -C --color=always .git}
\end{frame}

\begin{frame}[t]{Algunos de los ficheros en \fichero{.git}}

  \begin{overlayarea}{\textwidth}{2cm}
    \only<1>{
      \begin{block}{config}
        Contiene la configuración local.
      \end{block}
    }
    \only<2>{
      \begin{block}{description}
        Descripción corta textual para \paquete{gitweb}.
      \end{block}
    }
    \only<3>{
      \begin{block}{HEAD}
        Referencia simbólica a la revisión sobre la que estamos
        trabajando.
      \end{block}
    }
    \only<4>{
      \begin{block}{hooks}
        Manejadores de eventos. Ahora sólo tenemos ejemplos.
      \end{block}
    }
    \only<5>{
      \begin{block}{index}
        Contiene el área de preparación (la veremos después).
      \end{block}
    }
    \only<6>{
      \begin{block}{info/exclude (también .gitignore y/o global)}
        Patrones de ficheros a ignorar.
      \end{block}
    }
    \only<7>{
      \begin{block}{logs}
        Historial de las referencias: medida de seguridad.
      \end{block}
    }
    \only<8>{
      \begin{block}{refs}
        Referencias simbólicas a puntas de cada rama y etiquetas.
      \end{block}
    }
    \only<9>{
      \begin{block}{objects}
        Objetos, sueltos (gzip) o empaquetados (delta + gzip).
      \end{block}
    }
  \end{overlayarea}

  \begin{overlayarea}{\textwidth}{6cm}
    \only<1>{
      \orden{cat config}
      \showcommand{cat ejemplo/.git/config | sed -n '1,/logallrefupdates/p'}
    }
    \only<2>{
      \orden{cat description}
      \showcommand{cat ejemplo/.git/description}
    }
    \only<3>{
      \orden{cat HEAD}
      \showcommand{cat ejemplo/.git/HEAD}
    }
    \only<4>{
      \orden{ls hooks | head -5}
      \showcommand{ls --color=always ejemplo/.git/hooks | head -5}
    }
    \only<5>{
      \orden{git ls-files -s}
      \showcommand{cd ejemplo; git ls-files -s}
    }
    \only<6>{
      \orden{cat info/exclude}
      \showcommand{cat ejemplo/.git/info/exclude}
    }
    \only<7>{
      \orden{git reflog}
      \showcommand{cd ejemplo; git reflog}
    }
    \only<8>{
      \orden{ls -R refs}
      \showcommand{ls -CR --color=always ejemplo/.git/refs}
    }
    \only<9>{
      \orden{ls objects}
      \showcommand{ls -C --color=always ejemplo/.git/objects | ./chomp-last-newline.pl}
      \orden{ls objects/pack}
      \runcommand{ls -C ejemplo/.git/objects/pack}
      \orden{git gc}
      \runcommand{cd ejemplo; git gc}
      \orden{ls objects}
      \showcommand{ls -C --color=always ejemplo/.git/objects | ./chomp-last-newline.pl}
      \orden{ls objects/pack}
      \showcommand{ls -C --color=always ejemplo/.git/objects/pack | ./chomp-last-newline.pl}
    }
  \end{overlayarea}

\end{frame}

\begin{frame}{Área de preparación, caché o índice}
  \begin{block}{Concepto}
    Instantánea que vamos construyendo de la siguiente revisión.
  \end{block}

  \begin{block}{Diferencia entre Git y otros SCV}
    \begin{itemize}
    \item \inlinecmd{svn add} = añadir fichero a control de versiones
    \item \inlinecmd{git add} = añadir contenido a área de preparación
    \end{itemize}
  \end{block}

  \begin{block}{Consecuencias}
    \begin{itemize}
    \item Controlamos exactamente qué va en cada revisión
    \item Algo raro hasta acostumbrarse, pero es \emph{muy} potente
    \end{itemize}
  \end{block}
\end{frame}

\begin{frame}{Preparando revisiones (I)}
  \orden{echo "bueno" {>}> f.txt}
  \runcommand{cd ejemplo; echo "bueno" | tee -a f.txt}
  \orden{git add f.txt}
  \showcommand{cd ejemplo; git add f.txt}
  \orden{echo "malo" {>}> f.txt}
  \runcommand{cd ejemplo; echo "malo" | tee -a f.txt}
  \orden{echo "nuevo" > g.txt}
  \runcommand{cd ejemplo; echo "nuevo" | tee g.txt}
\end{frame}

\begin{frame}{Preparando revisiones (II)}
  \orden{git status}
  \showcommand{cd ejemplo; git status}
\end{frame}

\begin{frame}{Preparando revisiones (III)}
  \begin{columns}
    \column{.5\textwidth}
    \orden{git diff {-}-staged}\showcommand{cd ejemplo; git diff --staged}

    \column{.5\textwidth}
    \orden{git diff} \showcommand{cd ejemplo; git diff}
  \end{columns}
\end{frame}

\begin{frame}{Preparando revisiones (IV)}
  \orden{git commit -m "tercer commit"}
  \showcommand{cd ejemplo; git commit -m "tercer commit"}
  \orden{git status}
  \showcommand{cd ejemplo; git status}
\end{frame}

\begin{frame}{Preparando revisiones (V)}
  \begin{block}{Otras órdenes}
    \begin{itemize}
    \item \inlinecmd{git rm f}: retira del área de preparación y directorio de trabajo
    \item \inlinecmd{git rm {-}-cached f}: retira sólo del área de preparación
    \item \inlinecmd{git mv f g} = \inlinecmd{mv} + \inlinecmd{git rm} + \inlinecmd{git add}
    \item \inlinecmd{git add -p f}: seleccionamos trozos a preparar
    \item \inlinecmd{git add -i}: interfaz interactiva para preparar revisiones
    \end{itemize}
  \end{block}

  \begin{block}{Corregir la última revisión}
    \begin{itemize}
    \item \inlinecmd{git commit {-}-amend} corrige la última revisión
    \item No usar si se ha publicado la revisión ya a algún sitio
    \item Aun así, no se perdería información, pero es molesto de arreglar
    \end{itemize}
  \end{block}
\end{frame}

\subsection{Operaciones comunes}

\begin{frame}{Limpiar y deshacer}

  \begin{block}{Deshacer cambios}
    \begin{itemize}
    \item Del área de preparación:
      \inlinecmd{git reset f}, \inlinecmd{git reset}
    \item Del directorio de trabajo:
      \inlinecmd{git checkout f}, \inlinecmd{git checkout .}
    \item De revisiones anteriores ya publicadas: \inlinecmd{git
        revert commit} crea un \tipo{commit} inverso al especificado
    \end{itemize}
  \end{block}

  \begin{block}{Limpiar cosas fuera de control de versiones: \inlinecmd{git clean}}
    \begin{itemize}
    \item Simular qué haría: \inlinecmd{-n}
    \item Borrar directorios: \inlinecmd{-d}
    \item Borrar ignorados: \inlinecmd{-x}
    \item Realizar el borrado: \inlinecmd{-f} (\inlinecmd{-n} tiene mayor prioridad)
    \end{itemize}
  \end{block}

\end{frame}

\begin{frame}{Buscar cosas}
  \inlinecmd{git grep} es como \inlinecmd{grep -R *}, pero sólo busca
  entre los ficheros bajo control de versiones.

  \vfill

  \orden{git grep hola}
  \showcommand{cd ejemplo; git grep hola}
\end{frame}

\begin{frame}{Historial: por línea de órdenes (I)}
  \orden{git log}
  \showcommand{cd ejemplo; git log}
\end{frame}

\begin{frame}{Historial: por línea de órdenes (II)}
  \orden{git log {-}-pretty=oneline}
  \showcommand{cd ejemplo; git log --pretty=oneline}
\end{frame}

\begin{frame}{Historial: por línea de órdenes (III)}
  \orden{git log {-}-graph {-}-pretty=oneline {-}-decorate=short {-}-abbrev-commit}
  \showcommand{git log --graph --pretty=oneline --decorate=short --abbrev-commit 5f3c2bf..7eec99d}
\end{frame}

\begin{frame}{Historial: más cosas}
  \begin{block}{Opciones útiles}
    \begin{itemize}
    \item Por autor: \inlinecmd{{-}-author, {-}-committer}
    \item Por fecha: \inlinecmd{{-}-since, {-}-until}
    \item Por cambio: \inlinecmd{-S}
    \item Por mensaje: \inlinecmd{{-}-grep}
    \item Con parches: \inlinecmd{-p} (resumidos: \inlinecmd{--stat})
    \end{itemize}
  \end{block}

  \begin{block}{Otras órdenes}
    \begin{itemize}
    \item \inlinecmd{git show}: una revisión determinada
    \item \inlinecmd{git whatchanged}: estilo SVN
    \item \inlinecmd{gitk}: interfaz gráfica
    \item \inlinecmd{git instaweb}: interfaz Web
    \end{itemize}
  \end{block}
\end{frame}

\begin{frame}{Ayuda}
  \begin{block}{Listado de órdenes}
    \begin{itemize}
    \item \inlinecmd{git help} da un listado breve
    \item \inlinecmd{git help {-}-all} las lista todas (144+)
    \item Muchas son \enquote{fontanería}: sólo usamos la \enquote{porcelana}
    \end{itemize}
  \end{block}

  \begin{block}{Sobre una orden concreta}
    \begin{itemize}
    \item \inlinecmd{man git-orden}
    \item \inlinecmd{git help orden}
    \item \inlinecmd{git help -w orden} (en navegador)
    \item \inlinecmd{git orden -h}
    \end{itemize}
  \end{block}
\end{frame}

\section[Centralizado]{Flujo de trabajo centralizado}

\subsection{De Git a Git}

\begin{frame}{Trabajar con Git}

\end{frame}

\subsection{De Git a SVN}

\begin{frame}{Interoperar con SVN}

\end{frame}

\section[Distribuido]{Flujo de trabajo distribuido}

\begin{frame}{Colaborando entre varios repositorios}

\end{frame}

\section[Avanzado]{Aspectos avanzados}

\begin{frame}{Esculpir revisiones con add -i}

\end{frame}

\begin{frame}{Replantear ramas con rebase}

\end{frame}

\begin{frame}{Reorganizar ramas con rebase -i}

\end{frame}

\appendix

\begin{frame}{Fin de la presentación}
  \begin{center}
    {\Huge ¡Gracias por su atención!}

    \vspace{3em}

    {\Large
      \href{mailto:antonio.garciadominguez@uca.es}{antonio.garciadominguez@uca.es}}
  \end{center}
\end{frame}

\end{document}
