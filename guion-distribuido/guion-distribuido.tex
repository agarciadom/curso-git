%; whizzy -pdf

\documentclass[a4paper,11pt]{article}
\usepackage[latin1]{inputenc}
\usepackage[spanish]{babel}
\usepackage{amsthm}
\usepackage{graphicx}
\usepackage{hyperref}

\title{Gui�n de pr�cticas:\\Flujo de trabajo distribuido}
\author{Antonio Garc�a Dom�nguez}
\date{\today}

\RequirePackage{amsthm}

\newtheorem{pregunta}{Pregunta}
\newcounter{saveenum}
\newcommand{\savelp}{\setcounter{saveenum}{\value{enumi}}}
\newcommand{\loadlp}{\setcounter{enumi}{\value{saveenum}}}
\newenvironment{first-enumerate}{\begin{enumerate}}{\savelp{}\end{enumerate}}
\newenvironment{continue-enumerate}{\begin{enumerate}\loadlp{}}{\savelp{}\end{enumerate}}

%%% Local Variables: 
%%% mode: latex
%%% TeX-master: t
%%% End: 


\begin{document}

\maketitle
\begin{center}
  Distribuido bajo la licencia CC v3.0 BY-SA (\url{http://creativecommons.org/licenses/by-sa/3.0/deed.es}).

  \vskip 2em

  \includegraphics{../cc-by-sa}
\end{center}

\vskip .1\textheight

\tableofcontents{}

\clearpage

% m�s comunes:

% git blame (opciones: -f -n para ficheros movidos)
% git remote
% git fetch
% git pull
% git merge
% git push

% opciones alternativas en conectividad limitada:

% git bundle
% git format-patch + git am
% diff + git apply
% git send-email

%%% Para el pr�ximo gui�n:

% acceso r�pido por Web con gitweb
% git instaweb

% demonio Git
% git daemon

% gesti�n por Git de repositorios Git
% Gitosis (requiere openssh-server en Ubuntu)

% gesti�n de proyectos
% Redmine (�til usar Apache Mongrels y mongrel_cluster)

\end{document}

%%% Local Variables: 
%%% mode: latex
%%% TeX-master: t
%%% End: 
