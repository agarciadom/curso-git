%; whizzy -pdf

\documentclass[a4paper,11pt]{article}
\usepackage[latin1]{inputenc}
\usepackage[spanish]{babel}
\usepackage{amsthm}
\usepackage{tikz}
\usepackage{hyperref}
\usepackage{graphicx}
\usepackage[spanish]{varioref}

\title{Gui�n de pr�cticas:\\Gesti�n de ramas\footnote{Sin preguntas por el momento.}}
\author{Antonio Garc�a Dom�nguez}
\date{\today}

\RequirePackage{amsthm}

\newtheorem{pregunta}{Pregunta}
\newcounter{saveenum}
\newcommand{\savelp}{\setcounter{saveenum}{\value{enumi}}}
\newcommand{\loadlp}{\setcounter{enumi}{\value{saveenum}}}
\newenvironment{first-enumerate}{\begin{enumerate}}{\savelp{}\end{enumerate}}
\newenvironment{continue-enumerate}{\begin{enumerate}\loadlp{}}{\savelp{}\end{enumerate}}

%%% Local Variables: 
%%% mode: latex
%%% TeX-master: t
%%% End: 


\begin{document}

\maketitle
\begin{center}
  Distribuido bajo la licencia CC v3.0 BY-SA (\url{http://creativecommons.org/licenses/by-sa/3.0/deed.es}).

  \vskip 2em

  \includegraphics{../cc-by-sa}
\end{center}

\vskip .1\textheight

\tableofcontents{}

\clearpage

\section{Gesti�n b�sica de ramas}
\label{sec:creacion-ramas}

Las revisiones enviadas a un repositorio Git forman un grafo ac�clico
dirigido, en el que podemos tener l�neas de revisiones que se dividen
a partir de un cierto punto en m�ltiples ramas. Estas ramas
posteriormente pueden reunirse opcionalmente, aunque no es
estrictamente necesario.

Siempre estamos trabajando con una rama: por defecto, Git crea siempre
la rama \rama{master}, considerada normalmente como la rama principal
de desarrollo (\rama{trunk} para aquellos que conozcan
Subversion). Podemos ver en qu� rama estamos con:

\begin{lstlisting}
git branch
\end{lstlisting}

Obtendremos una salida como:

\begin{lstlisting}
* master
\end{lstlisting}

Esta salida indica que nos hallamos actualmente trabajando sobre la
punta de la rama \rama{master}, con lo que cualquier revisi�n que
vayamos enviando no s�lo crear� el objeto correspondiente, sino que
adem�s har� avanzar el puntero \rama{master} adem�s del
\commitish{HEAD}. Puede que no estemos sobre la punta de ninguna rama
si hemos movido el \commitish{HEAD} manualmente (despu�s veremos
c�mo). En dicho caso ver�amos algo as�:

\begin{lstlisting}
* (no branch)
  master
\end{lstlisting}

Hay que tener cuidado: si creamos nuevas revisiones sin que sean
alcanzables por una rama, estas revisiones no son alcanzables de forma
normal y ser�n recolectadas como basura tras un per�odo de gracia de
30 d�as por defecto. Podemos marcar la revisi�n actual como la punta
de una rama (sin llegar a cambiarnos a ella) con:

\begin{lstlisting}
git branch -a nombrerama
\end{lstlisting}

Por otro lado, podemos eliminar una rama (es decir, la referencia a su
punta) con:

\begin{lstlisting}
git branch -d nombrerama
\end{lstlisting}

Sin embargo, hay que tener cuidado en ciertos casos. Borrar la
referencia a una rama que ya ha sido reunida con alguna otra no tiene
problema, ya que sus revisiones son alcanzables desde la otra rama,
como se ve en el caso de \rama{develop}
en~\vref{fig:sin-problema-borrar-rama}. Sin embargo, si a�n no se ha
hecho esto, como en~\ref{fig:problema-borrar-rama}, podr�amos acabar
perdiendo las revisiones de dicha rama, al quedar inalcanzables por
toda referencia.

Por ello, \orden{git checkout -d develop} fallar�a en el segundo caso,
y si de verdad quisi�ramos eliminar esa rama y descartar todas sus
revisiones, sustituir�amos la opci�n \orden{-d} por \orden{-D}. Esto
es �til, por ejemplo, para descartar una rama que hayamos visto
improductiva por completo sin tener que reunirla.

\begin{figure}
  \centering
  \resizebox{\textwidth}{!}{\input{sinproblema-borrar-ramas}}
  \caption{Situaci�n no problem�tica al borrar la rama \rama{develop}}
  \label{fig:sin-problema-borrar-rama}
\end{figure}

\begin{figure}
  \centering
  \resizebox{\textwidth}{!}{\input{problema-borrar-ramas}}
  \caption{Situaci�n problem�tica al borrar la rama \rama{develop}}
  \label{fig:problema-borrar-rama}
\end{figure}

Se recuerda que para ver de forma c�moda el grafo de revisiones
desde la revisi�n actual, podemos usar \orden{gitk}, y para ver el
grafo completo, se puede utilizar \orden{gitk --all}.

\section{Cambio entre ramas}
\label{sec:cambio-entre-ramas}

% git checkout
% mencionar que podemos ir a algo que no sea la punta de una rama, pero los problemas que trae

% git stash (v1.5.3+)
% git stash apply
% git stash drop
% git stash pop
% git stash list
% git stash clear

% git checkout -m
% git checkout -b <commitish>

\section{Reuni�n de ramas}
\label{sec:reunion-ramas}

% git merge
% concepto de fast forward
%
% conflictos:
% * git diff --ours
% * git diff --theirs
% * git diff --base
% * resoluci�n
%
% git revert -m

\section{Reescribir varias revisiones en base a otra}
\label{sec:cambiar-rev-base-rama}

% git rebase
% git rebase --interactive

\end{document}

%%% Local Variables: 
%%% mode: latex
%%% TeX-master: t
%%% End: 
