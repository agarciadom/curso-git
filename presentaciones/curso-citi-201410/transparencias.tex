\documentclass[xcolor=svgnames]{beamer}

\mode<presentation>
{
  \usetheme{Warsaw}
  \setbeamertemplate{navigation symbols}{}
  \setbeamercovered{dynamic}
}

\usepackage[spanish,es-noshorthands]{babel}
\usepackage[utf8x]{inputenc}
\usepackage[T1]{fontenc}
\usepackage{csquotes}
\usepackage{tikz}
\usepackage{fancyvrb}
\usepackage[htt]{hyphenat}

\PrerenderUnicode{áéíóúÁÉÍÓÚçÇ}

\title[Introducción a Git]{Introducción al Sistema de Control de Versiones Distribuido Git}
\author{Antonio García Domínguez}
\date{16 octubre 2014}
\institute{Universidad de Cádiz \\\vspace{2em} \includegraphics{../../cc-by-sa}}

\AtBeginSection[]
{
  \begin{frame}<beamer>{Contenidos}
    \tableofcontents[currentsection,hideothersubsections]
  \end{frame}
}

\AtBeginSubsection[]
{
  \begin{frame}<beamer>{Contenidos}
    \tableofcontents[currentsection,subsectionstyle=show/shaded/hide]
  \end{frame}
}

\usetikzlibrary{calc,positioning,shapes,shapes.geometric}

% Semantic formatting
\newcommand*{\paquete}[1]{\texttt{#1}}
\newcommand*{\fichero}[1]{\textit{#1}}
\newcommand*{\rama}[1]{\structure{#1}}
\newcommand*{\tipo}[1]{\textit{#1}}
\newcommand*{\remoto}[1]{\structure{#1}}

% For formatting commands
\newcommand*{\inlinecmd}[1]{{\small\ttfamily\nohyphens{#1}}}
\newcommand*{\orden}[1]{{\scriptsize\ttfamily\$~\nohyphens{#1}\\}}

% For running Git commands and converting the ANSI escape sequences to LaTeX
\newcommand{\runcommand}[1]{\immediate\write18{#1}}
\newcommand{\showcommand}[2][cat]{%
  \runcommand{./err2out.sh '(#2)' | ansifilter -Lf | #1 | tee cmd.tmp}%
  {\scriptsize\ttfamily\input{cmd.tmp}}\runcommand{rm -f cmd.tmp}}
\newcommand{\runandshowcommand}[1]{
  \orden{#1}\showcommand{#1}}

% Required by ansifilter
\newcommand{\ws}[1]{\textcolor[rgb]{0,0,0}{#1}}

% For the diagrams talking about the existing types of VCS
\newenvironment{vcstypes}{
   \begin{tikzpicture}[
      node distance=6em,
      every path/.style={very thick},
      repo/.style={draw,rounded corners,fill=gray!50},
      wcopy/.style={draw,rounded corners,fill=green!20}]
}{\end{tikzpicture}}

\begin{document}

\begin{frame}
  \titlepage
\end{frame}

\begin{frame}{Contenidos}
  \tableofcontents[hideallsubsections]

  Materiales en \url{http://osl2.uca.es/wikiformacion/index.php/Git} y
  \url{http://gitorious.org/curso-git-osluca}.

  Una versión más antigua de la presentación (para Linux) está
  disponible en \url{http://goo.gl/2sCoK6}.
\end{frame}

\section{Introducción}

\subsection{Antecedentes}

\begin{frame}{Historia de los SCV}

  \begin{block}{Sin red, un desarrollador}
    \begin{description}
    \item[1972] Source Code Control System
    \item[1980] Revision Control System
    \end{description}
  \end{block}

  \begin{block}{Centralizados}
    \begin{description}
    \item[1986] Concurrent Version System
    \item[1999] Subversion (\enquote{CVS done right})
    \end{description}
  \end{block}

  \begin{block}{Distribuidos}
    \begin{description}
    \item[2001] Arch, monotone
    \item[2002] Darcs
    \item[2005] Git, Mercurial (hg), Bazaar (bzr)
    \end{description}
  \end{block}

\end{frame}

\begin{frame}{Historia de Git}

  \begin{block}{Antes de BitKeeper}
    Para desarrollar Linux, se usaban parches y \texttt{tar.gz}.
  \end{block}

  \begin{block}{BitKeeper}
    \begin{description}
    \item[02/2002] BitMover regala licencia BitKeeper (privativo)
    \item[04/2005] BitMover retira la licencia tras roces
    \end{description}
  \end{block}

  \begin{block}{Git}
    \begin{description}
    \item[04/2005] Linus Torvalds presenta Git, que ya reúne ramas
    \item[06/2005] Git se usa para gestionar Linux
    \item[02/2007] Git 1.5.0 es utilizable por mortales
    \item[09/2014] Última versión: Git 2.1.2
    \end{description}
  \end{block}

\end{frame}

\subsection{Tipos de SCV}

\begin{frame}{SCV centralizados}

  \begin{center}
    \begin{vcstypes}
      \draw[white] (3,2) rectangle (-3, -2);
      \node[repo]  (r)  {Repositorio central};

      \node<2->[wcopy,above right of=r] (w1) {Desarrollador A};
      \draw<2>[->,color=DarkGreen] (r) edge node[midway,right] {checkout} (w1);
      \draw<3>[<-,red] (r) edge node[midway,right] {commit} (w1);
      \draw<4->[-] (r) edge (w1);

      \node<2->[wcopy,below right of=r] (w2) {Desarrollador B};
      \draw<2>[->,color=DarkGreen] (r) edge node[midway,right] {checkout} (w2);
      \draw<3,5>[-] (r) edge (w2);
      \draw<4>[->,blue] (r) edge node[midway,right] {update} (w2);
    \end{vcstypes}
  \end{center}

  \begin{overprint}
    \onslide<1>
    \centering
    Tenemos nuestro repositorio central con todo dentro.

    \onslide<2>
    \centering
    Los desarrolladores crean \alert{copias de trabajo}.

    \onslide<3>
    \centering
    El desarrollador A manda sus cambios al servidor.

    \onslide<4>
    \centering
    El desarrollador B los recibe.

    \onslide<5>
    \centering ¿Y si se cae el servidor, o la red?
  \end{overprint}

\end{frame}

\begin{frame}{SCV distribuidos}

  \begin{center}
    \begin{vcstypes}
      \draw[white] (5,2) rectangle (-5, -2);

      \node[repo] (ra) at ( 0,-1) {Repositorio A};

      \node<2->[repo] (rb) at ( 3, 1) {Repositorio B};
      \draw<2>[->,color=DarkGreen] (ra) edge node[midway,right] {clone} (rb);
      \draw<3>[->,color=blue] (ra) edge node[midway,right] {pull} (rb);
      \draw<4>[<-,color=red] (ra) edge node[midway,right] {push} (rb);

      \node<5->[repo] (rc) at (-3, 1) {Repositorio C};
      \draw<5>[->,color=DarkGreen] (ra) edge node[midway,left] {clone} (rc);
      \draw<6>[<-,color=red] (ra) edge node[midway,left] {push} (rc);
      \draw<7>[<-,color=red] (ra) edge node[draw,fill=white,midway] {X} (rc);

      \draw<8>[->,color=red] (rc) edge node[midway,above] {push} (rb);

    \end{vcstypes}
  \end{center}

  \begin{overprint}
    \onslide<1> \centering Tenemos nuestro repositorio.

    \onslide<2> \centering Alguien \alert{clona} el repositorio.

    \onslide<3> \centering De vez en cuando se trae nuestros cambios recientes.

    \onslide<4> \centering De vez en cuando nos manda sus cambios.

    \onslide<5> \centering Viene otro desarrollador.

    \onslide<6> \centering Intenta hacer sus cambios locales...

    \onslide<7> \centering Pero no le funciona, o no tiene permisos para ello.

    \onslide<8> \centering Se los pasa al otro desarrollador sin más.

    \onslide<9> \centering La diferencia entre los repositorios es
    \emph{social}, no técnica.
  \end{overprint}

\end{frame}

\begin{frame}{Ventajas de un SCV distribuido (I)}

  \begin{block}{Rapidez}
    \begin{itemize}
    \item Todo se hace en local: el disco duro es más rápido que la
      red, y cuando esté todo en caché será más rápido aún
    \item Clonar un repositorio Git suele tardar \emph{menos} que
      crear una copia de trabajo de SVN, y ocupa menos
    \end{itemize}
  \end{block}

  \pause

  \begin{block}{Revisiones pequeñas y sin molestar}
    \begin{itemize}
    \item Nadie ve nada nuestro hasta que lo mandamos
    \item Podemos ir haciendo revisiones pequeñas intermedias
    \item Sólo mandamos cuando compila y supera las pruebas
    \item Podemos hacer experimentos de usar y tirar
    \end{itemize}
  \end{block}

\end{frame}

\begin{frame}{Ventajas de un SCV distribuido (II)}
  \begin{block}{Trabajo sin conexión}
    \begin{itemize}
    \item En el tren, avión, autobús, etc.
    \item Aunque no tengamos permisos de escritura
    \item Aunque se caiga la red, se puede colaborar
    \end{itemize}
  \end{block}

  \pause

  \begin{block}{Robustez}
    Falla el disco duro del repositorio bendito. ¿Qué hacer?
    \begin{itemize}
    \item Centralizado: copias de seguridad
    \item Distribuido: copias de seguridad y/o colaborar por otros medios
    \end{itemize}
  \end{block}
\end{frame}

\begin{frame}{Cuándo NO usar Git}

  \begin{block}{Git no escala ante muchos ficheros binarios}
    \begin{itemize}
    \item No sirve para llevar las fotos
    \item Ni para almacenar vídeos
    \end{itemize}
  \end{block}

  \begin{block}{Git no guarda metadatos}
    \begin{itemize}
    \item No guarda el dueño de los ficheros
    \item Sólo guarda si un fichero es ejecutable o no
    \item No sirve como sistema de copias de seguridad
    \end{itemize}
  \end{block}

\end{frame}

\section[Local]{Trabajo local}

\subsection{Primeras revisiones}

\begin{frame}{Instalación de Git en Windows}
  \begin{block}{Clientes libres}
    \begin{itemize}
    \item El original: \structure{Git for Windows} (\url{http://msysgit.github.io/})
      \begin{itemize}
      \item ``Git Bash'' permite usar todas las órdenes típicas en Linux
      \item ``Git GUI'' permite preparar y hacer revisiones de forma gráfica
      \end{itemize}
    \item Como TortoiseSVN: \structure{TortoiseGit} (\url{http://code.google.com/p/tortoisegit/})
      \begin{itemize}
      \item Para aprovechar toda la potencia de Git, hay que complementarlo con los de arriba
      \end{itemize}
    \end{itemize}
  \end{block}

  \begin{block}{Clientes privativos}
    \begin{itemize}
    \item \$79/persona: \structure{SmartGit} (\url{http://www.syntevo.com/smartgit/})
    \item Gratis por ahora: \structure{SourceTree} (\url{http://www.sourcetreeapp.com/})
    \end{itemize}
  \end{block}
\end{frame}

\begin{frame}{Configuración inicial: acceso a opciones}
  \begin{center}
    \includegraphics[width=\textwidth,height=.8\textheight,keepaspectratio]{tomas/configinicial-00-menu}
  \end{center}
\end{frame}

\begin{frame}{Configuración inicial: datos personales}
  \begin{center}
    \includegraphics[width=\textwidth,height=.8\textheight,keepaspectratio]{tomas/configinicial-01-pantalla}
  \end{center}
\end{frame}

\begin{frame}{Configuración inicial: gestor de credenciales}
  \begin{center}
    \includegraphics[width=\textwidth,height=.8\textheight,keepaspectratio]{tomas/configinicial-02-creds}
  \end{center}
\end{frame}

\begin{frame}{Creación de un repositorio: orden inicial}
  Sólo tenemos que crear un nuevo directorio y decirle a Git que cree
  un repositorio ahí.

  \begin{center}
    \includegraphics[width=\textwidth,height=.5\textheight,keepaspectratio]{tomas/crearrepo-00-menu}
  \end{center}
\end{frame}

\begin{frame}{Creación de un repositorio: repositorios ``pelados''}
  Un repositorio \emph{bare} no tiene un directorio de trabajo
  asociado: sólo lleva el histórico. Normalmente se usan en
  servidores, como en \url{des-sinf.uca.es}.

  Ahora queremos uno normal, por lo que dejamos la caja sin marcar.

  \begin{center}
    \includegraphics[width=\textwidth,height=.5\textheight,keepaspectratio]{tomas/crearrepo-01-bare}
  \end{center}
\end{frame}

\begin{frame}[t]{Nuestra primera revisión en <<master>>}

  \begin{overprint}
    \onslide<1>
    \begin{center}
      \includegraphics[width=\textwidth,height=.6\textheight,keepaspectratio]{tomas/primercommit-06-newfile}

      \vfill

      Creamos un fichero \fichero{f.txt} con ``hola''.
    \end{center}

    \onslide<2>
    \begin{center}
      \includegraphics[width=\textwidth,height=.6\textheight,keepaspectratio]{tomas/primercommit-00-add}

      \vfill

      Añadimos el fichero a control de versiones mediante ``Add''.
    \end{center}

    \onslide<3>
    \begin{center}
      \includegraphics[width=\textwidth,height=.6\textheight,keepaspectratio]{tomas/primercommit-01-dlgadd}

      \vfill

      Marcamos el fichero y pulsamos en ``OK''.
    \end{center}

    \onslide<4>
    \begin{center}
      \includegraphics[width=\textwidth,height=.6\textheight,keepaspectratio]{tomas/primercommit-02-success}

      \vfill

      Se nos confirma el añadido.
    \end{center}

    \onslide<5>
    \begin{center}
      \vspace{.15\textheight}

      \includegraphics[width=\textwidth,height=.6\textheight,keepaspectratio]{tomas/primercommit-03-commit}

      \vspace{.15\textheight}

      Usamos ``Commit'' para iniciar la creación de la revisión.
    \end{center}

    \onslide<6>
    \begin{center}
      \includegraphics[width=\textwidth,height=.6\textheight,keepaspectratio]{tomas/primercommit-04-dlgcommit}

      \vfill

      Introducimos el mensaje. En \url{des-sinf.uca.es}, no debemos olvidar poner \texttt{refs \#{}XYZ} al final de la primera línea.
    \end{center}

    \onslide<7-8>
    \begin{center}
      \includegraphics[width=\textwidth,height=.6\textheight,keepaspectratio]{tomas/primercommit-05-okcommit}

      \vfill

      Se nos confirma el commit. \alert{Ojo}: no se envía a ningún lado. Para eso existe ``Push'', que veremos después.
    \end{center}
  \end{overprint}

\end{frame}

\begin{frame}{Nuestra segunda revisión en <<master>>}

  \begin{overprint}
    \onslide<1>
    \begin{center}
      \includegraphics[width=\textwidth,height=.6\textheight,keepaspectratio]{tomas/segundocommit-00-changedfile}

      Añadimos una línea con ``adiós'' a \fichero{f.txt}.
    \end{center}

    \onslide<2>
    \begin{center}
      \includegraphics[width=\textwidth,height=.6\textheight,keepaspectratio]{tomas/segundocommit-01-emblem}

      Se cambia el emblema de \fichero{f.txt}, indicando que su contenido no coincide con el de la revisión actual.
    \end{center}

    \onslide<3>
    \begin{center}
      \includegraphics[width=\textwidth,height=.6\textheight,keepaspectratio]{tomas/segundocommit-02-dlgcommit}

      Creamos la segunda revisión con el mensaje ``segundo
      commit''. Podemos arreglar la última revisión (siempre que no la
      hayamos subido) o ver qué cambios hemos introducido.
    \end{center}

    \onslide<4>
    \begin{center}
      \includegraphics[width=\textwidth,height=.6\textheight,keepaspectratio]{tomas/segundocommit-03-emblemok}

      Se cambia el emblema de \fichero{f.txt}, indicando que su contenido sí coincide con el de la revisión actual.
    \end{center}
  \end{overprint}
\end{frame}

\runcommand{echo "hola" | tee ejemplo/f.txt}
\runcommand{cd ejemplo; git add f.txt}
\runcommand{cd ejemplo; git commit -m "primer commit"}
\runcommand{echo "adios" | tee -a ejemplo/f.txt}
\runcommand{cd ejemplo; git add f.txt}
\runcommand{cd ejemplo; git commit -m "segundo commit"}

\subsection{Conceptos}

%% MODELO DE DATOS

\begin{frame}{Estructura física de un repositorio Git}
  \begin{itemize}
  \item Directorio de trabajo: \fichero{ejemplo}
  \item Grafo de objetos: directorio oculto \fichero{ejemplo/.git}
  \item Área de preparación: \fichero{ejemplo/.git/index}
  \end{itemize}

  \begin{center}
    \includegraphics[height=.55\textheight]{tomas/modelodatos-00-gitbash}
  \end{center}
\end{frame}

\begin{frame}{Modelo de datos de Git}
  \begin{block}{Características}
    \begin{itemize}
    \item Un repositorio es un grafo orientado acíclico de objetos
    \item Hay 4 tipos de objetos: \tipo{commit}, \tipo{tree},
      \tipo{blob} y \tipo{tag}
    \item Los objetos son direccionables por contenido (resumen SHA1)
    \end{itemize}
  \end{block}

  \begin{block}{Consecuencias del diseño}
    \begin{itemize}
    \item Los objetos son inmutables: al modificarse, cambia su SHA1
    \item Git \emph{no} almacena información de ficheros
      movidos/copiados: los detecta automáticamente, así que no hay
      que mover/copiar de forma especial
    \item Git \emph{nunca} guarda más de un objeto una vez en el DAG,
      aunque aparezca en muchos sitios
    \end{itemize}
  \end{block}
\end{frame}

\begin{frame}{Modelo de datos de Git: revisiones (\tipo{commits})}
  \begin{block}{Contenido}
    \begin{itemize}
    \item Fecha, hora, autoría, fuente y un mensaje
    \item Referencia a revisión padre y a un \tipo{tree}
    \end{itemize}
  \end{block}

  \vfill

  \orden{git cat-file -p HEAD}
  \showcommand{cd ejemplo; git cat-file -p HEAD}
\end{frame}

\begin{frame}{Modelo de datos de Git: árboles (\tipo{trees})}
  \begin{block}{Contenido}
    \begin{itemize}
    \item Lista de \tipo{blobs} y \tipo{trees}
    \item Separa el nombre de un fichero/directorio de su contenido
    \item Sólo gestiona los bits de ejecución de los ficheros
    \item No se guardan directorios vacíos
    \end{itemize}
  \end{block}

  \vfill

  \orden{git cat-file -p HEAD:}
  \showcommand{cd ejemplo; git cat-file -p HEAD:}
\end{frame}

\begin{frame}{Modelo de datos de Git: ficheros (\tipo{blobs})}
  \begin{block}{Contenido}
    Secuencias de bytes sin ningún significado particular.
  \end{block}

  \vfill

  \orden{git cat-file -p HEAD:f.txt}
  \showcommand{cd ejemplo; git cat-file -p HEAD:f.txt}
\end{frame}

\begin{frame}{Modelo de datos de Git: etiquetas (\tipo{tags})}
  \begin{block}{Contenido}
    \begin{itemize}
    \item Referencias simbólicas inmutables a otros objetos
    \item Normalmente apuntan a \tipo{commits}
    \item Pueden firmarse mediante GnuPG, protegiendo la integridad de
      todo el historial hasta entonces
    \end{itemize}
  \end{block}

  \vfill

  \runcommand{cd ejemplo; git tag -a v1.0 -m "version 1.0" HEAD}
  \orden{git cat-file -p v1.0}
  \showcommand{cd ejemplo; git cat-file -p v1.0}
  \begin{tikzpicture}[remember picture,overlay]
    \tikzset{shift={(current page.center)},xshift=4cm,yshift=-2.2cm}
    \node (x) {\includegraphics[width=.4\textwidth]{tomas/creartag}};
  \end{tikzpicture}
\end{frame}

%%% MODELO DE DATOS

\begin{frame}[t]{Algunos de los ficheros en \fichero{.git}}

  \begin{overlayarea}{\textwidth}{2cm}
    \only<1>{
      \begin{block}{config}
        Contiene la configuración local.
      \end{block}
    }
    \only<2>{
      \begin{block}{HEAD}
        Referencia simbólica a la revisión sobre la que estamos
        trabajando.
      \end{block}
    }
    \only<3>{
      \begin{block}{hooks}
        Manejadores de eventos. Ahora sólo tenemos ejemplos.
      \end{block}
    }
    \only<4>{
      \begin{block}{index}
        Contiene el área de preparación (la veremos después).
      \end{block}
    }
    \only<5>{
      \begin{block}{info/exclude (también .gitignore y/o global)}
        Patrones de ficheros a ignorar.
      \end{block}
    }
    \only<6>{
      \begin{block}{logs}
        Historial de las referencias: medida de seguridad.
      \end{block}
    }
    \only<7>{
      \begin{block}{refs}
        Referencias simbólicas a puntas de cada rama y etiquetas.
      \end{block}
    }
    \only<8>{
      \begin{block}{objects}
        Objetos, sueltos (gzip) o empaquetados (delta + gzip).
      \end{block}
    }
  \end{overlayarea}

  \begin{overlayarea}{\textwidth}{6cm}
    \only<1>{
      \orden{cat config}
      \showcommand{cat ejemplo/.git/config | sed -n '1,/logallrefupdates/p'}
    }
    \only<2>{
      \orden{cat HEAD}
      \showcommand{cat ejemplo/.git/HEAD}
    }
    \only<3>{
      \orden{ls hooks | head -5}
      \showcommand{ls --color=always ejemplo/.git/hooks}
    }
    \only<4>{
      \orden{git ls-files -s}
      \showcommand{cd ejemplo; git ls-files -s}
    }
    \only<5>{
      \orden{cat info/exclude}
      \showcommand{cat ejemplo/.git/info/exclude}
    }
    \only<6>{
      \orden{git reflog}
      \showcommand{cd ejemplo; git reflog | head -5}
    }
    \only<7>{
      \orden{ls -R refs}
      \showcommand{ls -CR --color=always ejemplo/.git/refs}
    }
    \only<8>{
      \orden{ls objects}
      \showcommand{ls -C --color=always ejemplo/.git/objects | ./chomp-last-newline.pl}
      \orden{ls objects/pack}
      \runcommand{ls -C ejemplo/.git/objects/pack}
      \orden{git gc}
      \runcommand{cd ejemplo; git gc}
      \orden{ls objects}
      \showcommand{ls -C --color=always ejemplo/.git/objects | ./chomp-last-newline.pl}
      \orden{ls objects/pack}
      \showcommand{ls -C --color=always ejemplo/.git/objects/pack | ./chomp-last-newline.pl}
    }
  \end{overlayarea}

\end{frame}

\begin{frame}{Área de preparación, caché o índice}
  \begin{block}{Concepto}
    Instantánea que vamos construyendo de la siguiente revisión.
  \end{block}

  \begin{block}{``Add'' en herramientas para SVN y Git}
    \begin{itemize}
    \item \inlinecmd{svn add} = añadir fichero a control de versiones
      \begin{itemize}
      \item TortoiseSVN y TortoiseGit hacen esto
      \end{itemize}
    \item \inlinecmd{git add} = añadir contenido a área de preparación
      \begin{itemize}
      \item ``Git Bash'' y ``Git GUI'' hacen esto
      \end{itemize}
    \end{itemize}
  \end{block}

  \begin{block}{Pros y contras}
    \begin{itemize}
    \item Controlamos exactamente qué va (y qué \alert{no}) en cada revisión
    \item Algo raro hasta acostumbrarse
    \end{itemize}
  \end{block}
\end{frame}

\begin{frame}{Preparación de una revisión parcial}
  \begin{overlayarea}{\textwidth}{.65\textheight}
    \only<1>{
      \begin{center}
        \vspace{.125\textheight}

        \includegraphics[width=\textwidth,height=.4\textheight,keepaspectratio]{tomas/commitparcial-00-cambios}

        \vspace{.125\textheight}
      \end{center}
    }
    \only<2>{
      \begin{center}
        \vspace{.125\textheight}

        \includegraphics[width=\textwidth,height=.4\textheight,keepaspectratio]{tomas/commitparcial-01-menugitgui}

        \vspace{.125\textheight}
      \end{center}
    }
    \only<3>{
      \begin{center}
        \includegraphics[height=.6\textheight]{tomas/commitparcial-02-dlggitgui}
      \end{center}
    }
    \only<4>{
      \begin{center}
        \includegraphics[height=.6\textheight]{tomas/commitparcial-03-menustage}
      \end{center}
    }
    \only<5>{
      \begin{center}
        \includegraphics[height=.6\textheight]{tomas/commitparcial-04-efectostage}
      \end{center}
    }
    \only<6>{
      \begin{center}
        \includegraphics[height=.6\textheight]{tomas/commitparcial-05-msgcommit}
      \end{center}
    }
    \only<7>{
      \begin{center}
        \includegraphics[height=.6\textheight]{tomas/commitparcial-06-commitcreado}
      \end{center}
    }
    \only<8>{
      \begin{center}
        \includegraphics[height=.6\textheight]{tomas/commitparcial-07-restante}
      \end{center}
    }
  \end{overlayarea}
  \begin{overlayarea}{\textwidth}{3cm}
    \only<1>{
      \begin{center}
        Añadimos una línea al principio y al final de \fichero{f.txt}.
      \end{center}
    }
    \only<2>{
      \begin{center}
        Lanzamos ``Git GUI'' mediante el menú contextual del
        directorio actual.
      \end{center}
    }
    \only<3>{
      \begin{center}
        ``Git GUI'' nos muestra lo que no está aún en el área de
        preparación.
      \end{center}
    }
    \only<4>{
      \begin{center}
        Introducimos la línea añadida al principio de \fichero{f.txt}
        en el área de preparación.
      \end{center}
    }
    \only<5>{
      \begin{center}
        Ahora ya no sale esa línea como pendiente de añadir.
      \end{center}
    }
    \only<6>{
      \begin{center}
        Por el contrario, sale en la sección de ``preparado''. Creamos
        la revisión rellenando el mensaje y pulsando en ``Commit''.
      \end{center}
    }
    \only<7>{
      \begin{center}
        ``Git GUI'' nos confirma que la nueva revisión ha sido creada
        con éxito.
      \end{center}
    }
    \only<8>{
      \begin{center}
        Podemos volver a lo que está pendiente: son las líneas que no
        se introdujeron en la revisión anterior.
      \end{center}
    }
  \end{overlayarea}
\end{frame}

\runcommand{cd ejemplo; echo "bueno" | tee -a f.txt}
\runcommand{cd ejemplo; git add f.txt}
\runcommand{cd ejemplo; echo "malo" | tee -a f.txt}
\runcommand{cd ejemplo; echo "nuevo" | tee g.txt}
\runcommand{cd ejemplo; git commit -m "tercer commit"}

\subsection{Operaciones comunes}

\begin{frame}{Historial: ``Show Log''}
  \begin{center}
    \includegraphics[width=\textwidth,height=.7\textheight,keepaspectratio]{tomas/showlog.png}
  \end{center}

  Permite visualizar el histórico, buscar sobre él, obtener
  diferencias y operar sobre revisiones.
\end{frame}

\begin{frame}{Ver cambios desde la revisión actual: ``Diff...''}
  \begin{center}
    \includegraphics[width=\textwidth,height=.7\textheight,keepaspectratio]{tomas/diff.png}
  \end{center}

  Permite ver todos los cambios locales respecto a la revisión actual.
\end{frame}

\begin{frame}{Deshacer cambios: ``Revert...''}
  \begin{center}
    \includegraphics[width=\textwidth,height=.7\textheight,keepaspectratio]{tomas/revert.png}
  \end{center}

  Permite deshacer algunos o todos los cambios locales respecto a la
  revisión actual.
\end{frame}

\begin{frame}{Retirar ficheros inútiles: ``Clean Up...''}
  \begin{center}
    \includegraphics[width=\textwidth,height=.7\textheight,keepaspectratio]{tomas/cleanup.png}
  \end{center}

  Permite eliminar automáticamente los ficheros y/o directorios que no
  están bajo control de versiones.
\end{frame}

\section[Distribuido]{Trabajo distribuido}

\subsection{Manejo de ramas}

\runcommand{rm -rf sandbox-git}

\begin{frame}[fragile]{Clonar un repositorio}

  \begin{block}{Métodos de acceso}
    \begin{itemize}
    \item Git permite \url{git://}, SSH, HTTP(S) y \fichero{rsync}
    \item Nosotros: HTTPS con Apache + Redmine.pm + CGI de Git
    \end{itemize}
  \end{block}

  \begin{center}
    \includegraphics[width=\textwidth,height=.6\textheight,keepaspectratio]{tomas/clone}
  \end{center}

  \runcommand{git clone https://neptuno.uca.es/git/sandbox-git}
\end{frame}

\begin{frame}{Ramas en Git}
  \begin{block}{Concepto: líneas de desarrollo}
    \begin{description}
    \item[master]      Rama principal, equivalente a \rama{trunk}
    \item[develop]    Rama de desarrollo
    \item[nueva-cosa] Rama para añadir algo concreto
      (\foreignquote{english}{feature branch})
    \end{description}
  \end{block}

  \begin{block}{Diferencias con otros SCV}
    \begin{itemize}
    \item No son apaños con directorios, sino parte del modelo de datos
    \item Rama en Git: referencia mutable y compartible a una revisión
    \item Etiqueta en Git: referencia \emph{inmutable} a un objeto
    \end{itemize}
  \end{block}
\end{frame}

\begin{frame}{Ver ramas disponibles: ``Revision Graph''}
  \begin{center}
    \includegraphics[width=\textwidth,height=.6\textheight,keepaspectratio]{tomas/revgraph}
  \end{center}

  Muy útil para ver qué ramas hay y cómo se relacionan. Las que
  empiezan por ``origin'' son \emph{remotas}: representan al
  repositorio original y son de sólo lectura.
\end{frame}

\begin{frame}{Ver ramas disponibles: ``Show Log''}
  \begin{center}
    \includegraphics[width=\textwidth,height=.7\textheight,keepaspectratio]{tomas/gitlogramas}
  \end{center}

  Para ver todas las ramas, basta con marcar la caja ``All Branches''
  en la esquina inferior izquierda.
\end{frame}

\begin{frame}{Cambiando entre ramas: ``Switch/Checkout''}
  \begin{center}
    \includegraphics[width=\textwidth,height=.65\textheight,keepaspectratio]{tomas/checkout}
  \end{center}

  Podemos usar esta orden para cambiar a ramas locales o
  remotas. Normalmente, al cambiar a una rama remota es buena idea
  crear una rama local que la ``siga'', para poder trabajar nosotros
  en ella.
\end{frame}

\begin{frame}[t]{Reuniendo ramas: \foreignquote{english}{fast-forward}}
  \begin{overlayarea}{\textwidth}{.65\textheight}
    \only<1>{
      \begin{center}
        \includegraphics[width=\textwidth,height=.6\textheight,keepaspectratio]{tomas/checkoutej1}
      \end{center}
    }
    \only<2>{
      \begin{center}
        \includegraphics[width=\textwidth,height=.6\textheight,keepaspectratio]{tomas/mergeff}
      \end{center}
    }
    \only<3>{
      \begin{center}
        \includegraphics[width=\textwidth,height=.6\textheight,keepaspectratio]{tomas/mergeff-revgraph}
      \end{center}
    }
  \end{overlayarea}

  \begin{overlayarea}{\textwidth}{3cm}
    \only<1>{
      \begin{center}
        Creamos la rama \rama{ej1} a partir de
        \rama{origin/ejemplo-merge-master}.
      \end{center}
    }
    \only<2>{
      \begin{center}
        Reunimos la rama actual con \rama{origin/ejemplo-merge-ff}.
      \end{center}
    }
    \only<3>{
      \begin{center}
        No se ha creado ninguna revisión nueva: sólo hemos
        ``adelantado'' la punta de la rama.
      \end{center}
    }
  \end{overlayarea}
\end{frame}

\begin{frame}[t]{Reuniendo ramas: normal}
  \begin{overlayarea}{\textwidth}{.65\textheight}
    \only<1>{
      \begin{center}
        \includegraphics[width=\textwidth,height=.6\textheight,keepaspectratio]{tomas/checkoutej2}
      \end{center}
    }
    \only<2>{
      \begin{center}
        \includegraphics[width=\textwidth,height=.6\textheight,keepaspectratio]{tomas/mergenoff}
      \end{center}
    }
    \only<3>{
      \begin{center}
        \includegraphics[width=\textwidth,height=.6\textheight,keepaspectratio]{tomas/mergenoff-revgraph}
      \end{center}
    }
  \end{overlayarea}

  \begin{overlayarea}{\textwidth}{3cm}
    \only<1>{
      \begin{center}
        Creamos la rama \rama{ej2} a partir de
        \rama{origin/ejemplo-merge-master}.
      \end{center}
    }
    \only<2>{
      \begin{center}
        Reunimos la rama actual con \rama{origin/ejemplo-merge-noff}.
      \end{center}
    }
    \only<3>{
      \begin{center}
        Se ha creado ninguna revisión nueva con dos revisiones padre.
      \end{center}
    }
  \end{overlayarea}
\end{frame}

\begin{frame}{Reuniendo ramas: conflictos}
  \begin{overlayarea}{\textwidth}{.65\textheight}
    \only<1>{
      \begin{center}
        \includegraphics[width=\textwidth,height=.6\textheight,keepaspectratio]{tomas/merge-conflict}
      \end{center}
    }
    \only<2>{
      \begin{center}
        \includegraphics[width=\textwidth,height=.6\textheight,keepaspectratio]{tomas/merge-conflict-resolve}
      \end{center}
    }
    \only<3>{
      \begin{center}
        \includegraphics[width=\textwidth,height=.6\textheight,keepaspectratio]{tomas/merge-conflict-resolveopts}
      \end{center}
    }
    \only<4>{
      \begin{center}
        \includegraphics[width=\textwidth,height=.6\textheight,keepaspectratio]{tomas/merge-conflict-tortoisegitm}
      \end{center}
    }
    \only<5>{
      \begin{center}
        \includegraphics[width=\textwidth,height=.6\textheight,keepaspectratio]{tomas/merge-conflict-tortoisegitm-theirs}
      \end{center}
    }
    \only<6>{
      \begin{center}
        \includegraphics[width=\textwidth,height=.6\textheight,keepaspectratio]{tomas/merge-conflict-tortoisegitm-noconflicts}
      \end{center}
    }
    \only<7>{
      \begin{center}
        \includegraphics[width=\textwidth,height=.6\textheight,keepaspectratio]{tomas/merge-conflict-tortoisegitm-commit}
      \end{center}
    }
  \end{overlayarea}

  \begin{overlayarea}{\textwidth}{3cm}
    \only<1>{
        Tratamos de reunir \rama{ej3} (también sacada de
        \rama{origin/ejemplo-merge-master}) con
        \rama{origin/ejemplo-conflicto}, y nos da este error.
    }
    \only<2>{
        Usamos ``Resolve'' para que nos indique los conflictos a
        resolver.
    }
    \only<3>{
        El menú contextual sobre un fichero nos da las opciones
        disponibles: seleccionamos ``Edit conflicts''.
    }
    \only<4>{
        El editor integrado nos muestra nuestra versión (``Mine'') y
        la última en el repositorio (``Theirs''), con los conflictos
        en rojo. El menú contextual del conflicto nos deja
        decidir: elegimos usar ``Theirs''.
    }
    \only<5>{
        Así es como queda el fichero.
    }
    \only<6>{
        Pulsamos ``Save'' y se nos avisa de que no quedan más
        conflictos: marcamos el fichero como resuelto.
    }
    \only<7>{
        Ya podemos crear la nueva revisión. Es buena idea conservar el
        mensaje tal y como está, incluyendo el hecho de que hubo que
        resolver un conflicto.
    }
  \end{overlayarea}
\end{frame}

\subsection{Interacción con repositorios remotos}

\begin{frame}{Envío de objetos: ``Push''}
  \begin{center}
    \includegraphics[width=\textwidth,height=.65\textheight,keepaspectratio]{tomas/push}
  \end{center}
  
  Actualizamos una rama en un repositorio remoto (``remote'') a partir
  de una rama local. Puede subir todas las ramas de una vez si es
  necesario, o subir etiquetas (normalmente no se hace).
\end{frame}

\begin{frame}{Recepción de objetos: ``Fetch''}
  \begin{center}
    \includegraphics[width=\textwidth,height=.65\textheight,keepaspectratio]{tomas/fetch}
  \end{center}

  Actualizamos las ramas remotas (y posiblemente las etiquetas) a
  partir de los contenidos de un repositorio remoto.
\end{frame}

\begin{frame}{Recepción y reunión: ``Pull''}
  \begin{center}
    \includegraphics[width=\textwidth,height=.65\textheight,keepaspectratio]{tomas/pull}
  \end{center}
  
  Atajo de conveniencia: hace un ``Fetch'' y luego un ``Merge''.
\end{frame}

\begin{frame}{Sincronización: ``Sync''}
  \begin{center}
    \includegraphics[width=\textwidth,height=.65\textheight,keepaspectratio]{tomas/sync}
  \end{center}
  
  Específico de TortoiseGit: permite comparar una rama local con una
  remota y ver qué podemos mandar (``Out Commits'') o recibir (``In
  Commits''). Tiene atajos para lanzar los verbos anteriores.
\end{frame}

\begin{frame}{Flujo de trabajo centralizado}

  \begin{block}{Secuencia típica: muy similar a SVN}
    \begin{itemize}
    \item Creamos un clon del repositorio \emph{dorado}
    \item Nos actualizamos con \inlinecmd{git pull}
    \item Si hay conflictos, los resolvemos
    \item Hacemos nuestros cambios sin preocuparnos mucho de Git
    \item Los convertimos en revisiones cohesivas y pequeñas
    \item Los enviamos con \inlinecmd{git push}
    \end{itemize}
  \end{block}

\end{frame}

\begin{frame}{Flujos de trabajo distribuidos: variantes}

  \begin{block}{Un integrador}
    \begin{itemize}
    \item Cada desarrollador tiene rep. privado y rep. público
    \item Los desarrolladores colaboran entre sí
    \item El integrador accede a sus rep. públicos y actualiza el
      repositorio oficial
    \item Del repositorio oficial salen los binarios
    \item Los desarrolladores se actualizan periódicamente al oficial
    \end{itemize}
  \end{block}

  \begin{block}{Director y tenientes}
    \begin{itemize}
    \item El integrador (dictador) es un cuello de botella
    \item Se ponen intermediarios dedicados a un subsistema (teniente)
    \end{itemize}
  \end{block}

\end{frame}

\appendix

\begin{frame}{Fin de la presentación}
  \begin{center}
    {\Huge ¡Gracias por su atención!}

    \vspace{3em}

    {
      \Large
      \href{mailto:antonio.garciadominguez@uca.es}{antonio.garciadominguez@uca.es}

      \vspace{1em}

      @antoniogado\_es
    }
  \end{center}
\end{frame}

\end{document}
