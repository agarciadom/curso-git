\RequirePackage{amsthm}
\RequirePackage{listings}
\RequirePackage{tikz}
\RequirePackage{verbatim}

\newtheorem{pregunta}{Pregunta}

% no se puede usar el atajo <<x>> de Babel dentro de una orden: en su
% lugar, invocamos manualmente al entorno quoting

\newcommand*{\commitish}[1]{\texttt{#1}}
\newcommand*{\fichero}[1]{\texttt{#1}}
\newcommand*{\gitk}{\texttt{gitk}}
\newcommand*{\orden}[1]{\texttt{#1}}
\newcommand*{\preambulo}[1]{\texttt{#1}}
\newcommand*{\rama}[1]{\texttt{#1}}
\newcommand*{\remote}[1]{\begin{quoting}#1\end{quoting}}
\newcommand*{\repositorio}[1]{\texttt{#1}}
\newcommand*{\treeish}[1]{\texttt{#1}}
\newcommand*{\usuario}[1]{\begin{quoting}#1\end{quoting}}
\newcommand*{\variable}[1]{\emph{#1}}

\newcommand*{\inputscalew}[2]{\resizebox{#1}{!}{\input{#2}}}

\newcounter{saveenum}
\newcommand{\savelp}{\setcounter{saveenum}{\value{enumi}}}
\newcommand{\loadlp}{\setcounter{enumi}{\value{saveenum}}}
\newenvironment{first-enumerate}{\begin{enumerate}}{\savelp{}\end{enumerate}}
\newenvironment{continue-enumerate}{\begin{enumerate}\loadlp{}}{\savelp{}\end{enumerate}}

% Comentar o descomentar en función de que queramos las notas o no
%\newenvironment{nota}{\begin{quote}\emph{Nota: }}{\end{quote}}
\newenvironment{nota}{\comment\emph{Nota: }}{\endcomment}

\lstset{basicstyle=\small,breaklines=true,columns=fullflexible,frame=tb}

\newenvironment{tikzpicture-fixed}{\begin{tikzpicture}\shorthandoff{<>}}{\shorthandon{<>}\end{tikzpicture}}

%%% Local Variables: 
%%% mode: latex
%%% TeX-master: t
%%% End: 
